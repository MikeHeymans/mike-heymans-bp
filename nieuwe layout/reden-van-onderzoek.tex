\section{Stand van zaken}
\label{sec:onderzoeksvragen}
	Adobe Experience Manager is het gelicentieerde  contentmanagementsysteem (CMS) van Adobe. 
	Een CMS-applicatie maakt het mogelijk voor mensen zonder kennis van HTML of CSS om webpagina’s aan te maken of te wijzigen. 
	Het is ideaal voor websites die vaak de inhoud van hun site wijzigen of snel willen inspelen op actuele gebeurtenissen. 
	Dit komt doordat designers en marketeers zelf de pagina's kunnen editeren in plaats van eerst content uit te denken en de realisatie over te laten aan een front-end developer. 
	\par
	Toen AS Adventure besloot om zijn huidige website, en die van zijn zusterbedrijven, niet langer uit te besteden aan een extern bedrijf maar deze intern te gaan beheren en ontwikkelen, is er besloten om deze met het AEM-platform op te zetten. 
	Dit zou het bedrijf in staat stellen zijn content volledig door de designers te laten beheren, zonder tussenkomst van developers.
	Denk maar aan de homepage die van achtergrondafbeelding wijzigt of een gepersonaliseerde pagina voor een merk, allemaal mogelijk dankzij AEM.
	\par
	Dat dit allemaal out-of-the-box mogelijk is, is te mooi om waar te zijn. Voor het content team aan de slag kan, moeten er componenten gebouwd worden, dit is het werk van de ontwikkelaars. Eigenlijk komt het erop neer dat componenten de bakstenen vormen om een website op te bouwen en de designers zijn de metsers. Ze kunnen lustig huisjes bouwen, maar wanneer er nood is aan een baksteen met een nieuwe feature moet hiervoor een aanvraag gedaan worden bij de steenbakkers, gespeeld door de ontwikkelaars. Deze stenen bevatten niet enkel HTML en CSS maar ook Java (en diverse frameworks) zijn nodig om dit nieuw type steen te ontwikkelen. De designers krijgen de nieuwe baksteen en kunnen hiermee aan de slag gaan.
	\par
	Buiten de content die door het marketingteam wordt voorzien is er ook data die vanuit een databank moet komen. 
	Hierbij hebben we het over records die niet manueel worden aangepast maar met duizenden tegelijk of data die niet enkel betrekking hebben op de site maar ook tijdens andere processen een rol spelen. 
	Een voorbeeld hiervan zijn de prijzen, tijdens de solden is het niet re\"eel om elke prijs handmatig aan te passen op de pagina’s. 
	In het onwaarschijnlijke geval dat het handmatig aanpassen van een specifieke prijs nodig is, zou deze data enkel op de pagina gewijzigd zijn en niet in het ERP. 
	Het is dus logischer dat het ERP deze wijziging doet en door duwt naar de pagina. Ook hier spelen de ontwikkelaars een rol, het is hun taak de systemen die dit mogelijk maken te bedenken en te realiseren.
	\par
	Toen AS Adventure aan dit project begon waren er processen uitgedacht om deze wijzigingen weer te geven en toen de eerste twee sites, Juttu en Yaya, live stonden was iedereen tevreden over het resultaat. 
	Maar deze shops zijn relatief nieuw en beperkt in aanbod, toen de vernieuwde AS Adventure site live ging staken er enkele problemen de kop op. Aangezien deze problemen in productie voorkwamen was wachten geen optie en moesten deze opgelost worden, niet zozeer met de beste maar wel de snelste oplossing.
	\par
	\section{Onderzoeksvragen}
	Deze proef heeft als nut het herzien van wat er tot nu toe opgebouwd werd, hoe het AEM-platform aangepast is geweest om aan de behoeften te voldoen en of dit ook effectief de beste manier is om de vereiste functionaliteit te bekomen. 
    Door het in kaart brengen van onze methodologi\"en hopen we een dieper inzicht te verwerven in het platform alsook een leidraad te voorzien voor ontwikkelaars die een eerste keer kennis maken met AEM of zij die reeds werken met AEM en geconfronteerd worden met de problemen waarvoor wij een oplossing zoeken. 
	Zoals het gezegde luidt: het is goed te leren van je fouten maar beter om te leren uit die van een ander.
