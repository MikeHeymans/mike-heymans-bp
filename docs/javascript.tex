% !TeX spellcheck = nl_NL
\documentclass{article}

\begin{document}
	\subsection{Javascript}
     De laatste methode is een oude bekenden, hierover wordt er niet teveel uitgeweken. Er wordt verwacht dat de lezer reeds bekend is met Javascript en moest dit niet het geval zijn, er is een overvloed aan tutorials. Het is toch belangrijk te vermelden dat niet alle dynamische factoren vanuit het AEM platform verzorgt kunnen worden. Tijdens de ontwikkeling van een applicatie zal het nu eenmaal mogelijk moeten zijn om te reageren op acties van de gebruiker. Als een gebruiker van productmaat wisselt, willen we ook de prijs en levertermijn aanpassen zonder de rest van de pagina te herladen. Omdat AEM zich focust op het afhandelen van HTML request, zal hierbuiten voor een oplossing gezocht moeten worden. Deze oplossing wordt gevonden in Javascript (of een hierop gebaseerd framework) en vult een applicatie aan waar AEM dat niet kan.
\end{document}