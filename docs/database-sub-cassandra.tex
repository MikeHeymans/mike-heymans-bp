% !TeX spellcheck = nl_NL
\documentclass{article}

\begin{document}
	\subsection{Apache Cassandra}
	Apache Cassandra is een opensource, NoSql (Not only Sql) database die de dag van vandaag door vele internationale bedrijven 
	in gebruik is genomen. 
	Netflix, Instagram en het CERN hebben in Cassandra een performante, stabiele databank gevonden, 
	die de beschikbaarheid van hun data garandeert.
	\par
	Deze garantie wordt geboden door de structuur waarmee Cassandra data beheert. 
	Een databank bestaat uit een collectie van noden die een cluster vormen. 
	Het beheer gebeurt decentraal, elke node heeft dezelfde rol, 
	dit maakt dat wanneer een node offline is, de andere ongehinderd blijven werken. 
	Om het volle potentieel van Cassandra te benutten, is het aangeraden om de noden niet enkel over verschillende servers 
	maar ook over verschillende datacenters te verspreiden. 
	Bij deze setup mag er een datacenter offline gaan, de gebruikers zullen hier geen weet van hebben.  
	En wanneer het datacenter terug online is, zal de data gedeeld worden met de herstelde noden.
	Dit is ook positief voor de schaalbaarheid. 
	Wanneer een node wordt toegevoegd, haalt deze zijn configuratie op bij een bestaande node en sluit zich naadloos aan in de cluster.
	\par
	Bij het aanmaken van een cluster kan men een replicatie factor meegeven. 
	Deze geeft aan op hoeveel verschillende noden een record moet bewaard worden. 
	Bij een factor 3 zal elk record op 3 noden beschikbaar zijn. 
	Des te groter de factor, des te groter de performantie en stabiliteit maar ook de duplicatie aan data vergroot mee.
	Bij het kiezen van de replicatiefactor moet er een afweging gemaakt worden tussen niveau van beschikbaarheid en
	het extra geheugen dat de replicatie vereist.
	\par
	Een groot verschil met een relationele databank is dat het linken van verscheidene tabellen niet mogelijk is.
	In plaats daarvan werkt Cassandra met collecties die rechtstreeks in een kolom worden opgeslagen. 
	Dit wordt mogelijk door het aanmaken van UDTs (User Defined Types). 
	Wanneer een lijst van UDTs (of primitieve types) wordt opgeslagen, wordt deze omgevormd naar een json-achtige structuur. 
	Doordat alle data in één rij zitten, hoeven er geen joins gedaan te worden over meerdere tabellen 
	wat de performantie aanzienlijk verhoogt.
	\par
	Er zijn ook enkele nadelen verbonden aan het kiezen voor Cassandra als database, één van deze nadelen is dat het, out-of-the-box,
	onmogelijk is om tabellen te querieën zoals je met een relationele databank zou doen. 
	Dit komt doordat Cassandra werkt met key-value paren om de data op te slagen en dus enkel aan de hand van een key 
	(of deel van een key) naar een record gezocht kan worden. 
	Een bijgevolg hiervan is dat aggregatie functies enkel op partitie niveau kunnen worden uitgevoerd.
	\par
	Aangezien Cassandra een non-relationele databank is, is een ander nadeel de afwezigheid van transacties op databank niveau.
	Enkel op rijniveau kan men gebruik maken van lightweight transacties door een conditie toe te voegen voor een schrijf operatie.
	Hiervoor zou men bv. een version id kunnen toevoegen aan een tabel zodat wanneer men een schrijf operatie wil doen,
	er gekeken kan worden of de versies nog matchen. 
	Indien dit niet het geval is wordt de operatie niet uitgevoerd en de gebruiker hiervan op de hoogte gebracht.
\end{document}