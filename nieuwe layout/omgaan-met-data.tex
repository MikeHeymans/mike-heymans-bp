% !TeX spellcheck = nl_NL
\documentclass{article}

\begin{document}
	\section{Omgaan met data}
	Nu we kennis hebben van AEM wordt het tijd dat we mogelijkheden ontdekken met betrekking tot onbeheerde data, hiermee spreken we over de zaken die we niet via AEM wijzigen of zelfs niet in AEM opslaan. De meeste organisaties slagen hun data op in een ERP-applicatie en verwachten dat de wijzigingen van deze data doorkomen op de website. We bespreken vier manieren hoe we dit kunnen verwezenlijken, welk ontwikkelingswerk moet gebeuren en wat de voor-en nadelen zijn.
	
	\subsection{JavaScript} 
	De eerste manier is eentje die iedere ontwikkelaar al heeft gebruikt.
\end{document}