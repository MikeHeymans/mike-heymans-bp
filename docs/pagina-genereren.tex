% !TeX spellcheck = nl_NL
\documentclass{article}

\begin{document}
	\subsection{Pagina's genereren} 
	\subsubsection{Wat is het?}
	Als we pagina's gaan genereren, gaan we wijzigingen van het ERP doorduwen naar AEM. AEM gaat aan de hand van deze informatie noden aanmaken in onze JCR repository en per taalvariant een pagina voorzien. Eenmaal de pagina's aangemaakt zijn, kunnen ze via de corresponderende url worden opgehaald.
	\subsubsection{Hoe werkt het?}
	Om deze optie te voorzien moeten er enkele zaken gecodeerd worden in Java, met name de functionaliteit om via de SlingRepository de data weg te schrijven naar een node en de PageManager van AEM gebruiken om de pagina's aan te maken. Alsook moeten we een HTML template voorzien die we kunnen gebruiken om onze categorie weer te geven. Tijdens het genereren van de pagina kunnen we verwijzen naar deze template om een standaard look te geven aan de pagina. Eenmaal deze pagina er is kan het contentteam aan de slag om deze een persoonlijke touch te geven.
	\par
	Om een node aan te maken moeten we eerst een Session verkrijgen waarin we kunnen werken, dit kunnen we doen door de login methode op onze SlingRepository aan te roepen met onze credentials. JcrUtils is een voorziene klasse waarmee we noden kunnen aanmaken, hetgeen we nodig hebben is een pad waaronder we deze willen opslaan, het type dat we aan de node willen toekennen en onze reeds verkregen sessie. Als we de node hebben aangemaakt rest ons nog het over mappen van de velden naar onze noden. Indien ons model velden bevat die een klasse op zich zijn, moeten we met geneste noden werken. Dit nesten heeft geen beperkingen en kan zo diep als nodig zijn. Het is aan te raden om deze velden te voorzien met een prefix om een duidelijk onderscheid te maken tussen de properties die we zelf cre\"eren en diegene die door JCR aangemaakt zijn. Stel dat we een veld 'title' hebben die we wensen op te slaan, doen we dit onder 'jouw-tag:title'. Wanneer de velden zijn overgezet mogen we niet vergeten om de save methode van onze sessie aan te roepen zodat onze data gepersisteerd wordt.
	\par
	De volgende stap is een pagina voor ons model aanmaken, dit per taal waarin onze website beschikbaar is. Een pagina aanmaken gebeurt op een gelijkaardige manier als een node. We gebruiken de PageManager die de AEM API ons beschikbaar stelt. Omdat we zaken gaan persisteren moeten we ook hier werken met een sessie. Via de pageManager kunnen we onze root pagina van de website opvragen. De directe kinderen van deze pagina stellen de talen voor waarin onze site beschikbaar is. We itereren over deze lijst waarbij we per kind (lees taal) een pagina genereren. We maken deze pagina niet rechtstreeks onder de taalpagina aan maar voorzien een tussen pagina, bv. categorie\"en. Wanneer we een categorie met ID 14 hebben gaan we de pagina hiervoor opslaan onder "jouw-site"/nl/categorieen/14. Het pad dat we dan meegeven aan de create methode van de pageManager is het pad van de ouder waaronder we een pagina willen maken. Naast deze parameter geven we ook de pagina naam mee (in ons voorbeeld 14), de template (de locatie van de HTML die we gebruiken om onze pagina te tonen) en een modeltitel (bv. 'category'). Nu de pagina is aangemaakt kunnen we hiervan de node opvragen en kunnen we, conform aan vorige paragraaf, deze voorzien van properties. Ook hier prefixen we best onze namen.
    \par
    In vorige paragraaf gebruiken we een template om onze categorie te tonen, deze gaan we zelf voorzien. Het is best om hier gebruik te maken van het componenten systeem waarvoor AEM gekend is. In plaats van een specifieke html te coderen gaan we deze opsplitsen in herbruikbare componenten. Hoe we componenten bouwen hebben we reeds gezien in een vorig hoofdstuk.
    \par
    Eenmaal we de nodige componenten hebben aangemaakt kunnen deze gebruikt worden om een template te voorzien. We kunnen de componenten rechtstreeks aan onze template toevoegen of een tussen component voorzien die kleinere componenten combineert. Dit kan handig zijn wanneer dezelfde combinatie van componenten meermaals gebruikt wordt (denk maar aan een navigatie), ook de CSS en JS bestanden kunnen we in deze wrapper inladen zodat we ons daarover geen zorgen moeten maken wanneer we templates samenstellen. Eenmaal de template af is kan deze gebruikt worden om automatisch pagina's te genereren of om via de author handmatig een pagina toe te voegen.
	\subsubsection{De voordelen}
    Het voordeel van deze manier van werken is dat wanneer een gebruiker de pagina opvraagt, deze reeds voorzien is van data waardoor er geen repositories worden aangesproken. Uiteraard is dit wel het geval als we componenten toevoegen die dynamische content voorzien maar de properties die we aan onze node hebben toegevoegd zijn reeds aanwezig.
    \par
     Een bijkomend voordeel is dat, door het feit dat er een pagina aangemaakt is, we deze via de author kunnen bewerken. We maken weliswaar een pagina via een standaard template maar hier hoeft het niet bij te blijven. We kunnen deze via de author gaan editeren en van specifieke content zoals een achtergrond afbeelding, een paragraaf met extra informatie of een uitgelicht product voorzien. Pagina's genereren is de enigste manier van werken die dit mogelijk maakt.
	\subsubsection{De nadelen}
    Het grote nadeel van deze methode is dat dit een vrij intensieve operatie is en het genereren zelden beperkt is tot \'e\'en enkele pagina, wanneer een site drie talen heeft zal AEM telkens drie pagina's aanmaken. Dit wordt nog eens ge\"expandeerd indien men meerdere versies van een site heeft. Stel dat we een .com versie hebben met drie talen, een .be versie met drie talen en een .nl versie met twee talen. Dit betekent dat per categorie AEM zeven pagina's moeten voorzien. Voor een model waar relatief weinig wijzigingen aan gebeurt is dit doenbaar, zoals ons categorie voorbeeld. Maar als we dit willen doortrekken naar producten, waar er mogelijks duizenden per dag wijzigen, is dit niet houdbaar. Onze author zou continu belast worden met het genereren van pagina's en potentieel bezwijken onder de load. Een tweede author plaatsen is zelden een optie omdat dit een tweede licentie vereist wat de kost zou verdubbelen. Zelfs als het budget dit toelaat zou er ontwikkelingen moeten gebeuren om de authors synchroon te houden. 
\end{document}