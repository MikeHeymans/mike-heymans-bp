% !TeX spellcheck = nl_NL
\documentclass{article}

\begin{document}
	\section{Conclusie}
    \subsection(Adobe Experience Manager)
    AEM doet wat het moet doen, het biedt een schaalbare oplossing voor organisaties die op zoek zijn naar een CMS applicatie. 
    De UI van de editor zorgt ervoor dat marketeers, zonder al te veel moeite, aan de slag kunnen met het maken, wijzigen en publiceren van pagina's. 
    Doordat de leercurve zo laag is, zijn werknemers sneller ingewerkt waardoor er minder ge\"Investeerd moet worden in opleidingen.
    \par
    De leercurve voor ontwikkelaars ligt des te hoger, dit komt omdat AEM bestaat uit een combinatie van verschillende frameworks. 
    De kans dat een ontwikkelaar reeds met al deze frameworks te maken heeft gehad (indien deze nog niet met AEM gewerkt heeft) is nihil.
    Vooral OSGi is een leerprocess waar redelijk veel aandacht moet besteet worden.
    \par
    Ook al is de editor intu\"itief in de omgang met componenten, het maken van deze componenten is iets ingewikkelder. Voornamelijk via een Maven project
    is het een kwestie van vallen en opstaan. Tijdens het leerprocess was \'e\'en van de uitdagingen om bruikbare documentatie hierover te vinden. 
    Het maken van componenten via de author is beter gedocumenteerd en kan gebruikt worden om de gedachtengang beter te begrijpen. 
    Toch is het even schrikken wanneer je je eerste (template) project opent en alle modules en packages ziet verschijnen.
    Zonder begeleiding met kennis hierover is het begrijpen van de onderdelen een moeilijke/frustrerende taak.
    \par
    AEM is de CMS applicatie voor bedrijven die geen schrik hebben om te investeren in de opzet ervan, het aannemen van ervaren developers is een must.
    De vrijheid dat AEM biedt met betrekking tot het beheren van een grote hoeveelheid content en de out-of-the-box ondersteuning voor meerdere sites maakt
    het voor gewenst bij de internationale commerci\"ele bedrijven. 
    \subsection(Dynamische Data)
\end{document}