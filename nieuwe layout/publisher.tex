
	\subsection{De publisher}
	\subsubsection{Het nut van een publisher}
	Een publisher is niet meer dan een afgeslankte author, alle management en editor tools zijn weggenomen en enkel de repositories worden bijgehouden. Dit zorgt dat marketeers geen pagina's kunnen editeren op een publisher maar dat deze wel in staat is om pagina's te genereren. Normaal zou de author voor elk request een pagina generen maar dit kan uitbesteed worden aan één of meer publishers, zo kan de author zich focussen op het content management gedeelte. Het voorzien van meerdere publishers voorkomt ook een \textquotedbl Single Point Of Failure\textquotedbl{}, als een publisher crasht, blijft de site draaien zolang deze gebruik kan maken van een andere instantie.
	\par
	Voor een publisher een request kan afhandelen moet deze weet hebben van de content op de website. Hiervoor heeft een publisher een kopie van alle noden in een persoonlijke JCR-repository. Elke publisher is bij de author geregistreerd als een agent, wanneer er een pagina opnieuw wordt gepersisteerd wordt deze pagina naar elke publisher verzonden zodat deze up-to-date blijven met de author. Een binnenkomend request heeft hierdoor altijd hetzelfde resultaat ongeacht de publisher die deze verzorgt.
	\subsubsection{Een publisher opzetten}
	Om een publisher op te zetten gaan we hetzelfde te werk als bij de author, we maken een folder \textquotedbl publish\textquotedbl{} met wederom onze jar en licentie in, de jar moeten we hernoemen. We vervangen \textquotedbl author\textquotedbl{} door \textquotedbl publish\textquotedbl{} en de poort passen we aan naar 4503 wat resulteert in het volgende commando.	
	\begin{lstlisting}
		 $ java -XX:MaxPermSize=256m -Xmx1024M -jar cq5-publish-p4503.jar
	\end{lstlisting}
	 Dit start een publisher op poort 4503, het verschil is dat deze geen editor ter beschikking stelt, we kunnen enkel onze website opvragen.
